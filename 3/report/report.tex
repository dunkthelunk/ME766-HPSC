%----------------------------------------------------------------------------------------
%	PACKAGES AND OTHER DOCUMENT CONFIGURATIONS
%----------------------------------------------------------------------------------------

\documentclass{article}

\usepackage{booktabs}  % nice looking tables
\usepackage{siunitx}
\usepackage{fancyhdr} % Required for custom headers
\usepackage{lastpage} % Required to determine the last page for the footer
\usepackage{extramarks} % Required for headers and footers
\usepackage[usenames,dvipsnames]{color} % Required for custom colors
\usepackage{graphicx} % Required to insert images
\usepackage{listings} % Required for insertion of code
\usepackage{libertine} % Required for the courier font
\usepackage{inconsolata}
% Margins
\topmargin=-0.45in
\evensidemargin=0in
\oddsidemargin=0in
\textwidth=6.5in
\textheight=9.0in
\headsep=0.25in

\linespread{1.1} % Line spacing

% Set up the header and footer
\pagestyle{fancy}
\lhead{\hmwkAuthorName} % Top left header
\chead{\hmwkClass : \hmwkTitle} % Top center head
\rhead{\firstxmark} % Top right header
\lfoot{\lastxmark} % Bottom left footer
\cfoot{} % Bottom center footer
\rfoot{Page\ \thepage\ of\ \protect\pageref{LastPage}} % Bottom right footer
\renewcommand\headrulewidth{0.4pt} % Size of the header rule
\renewcommand\footrulewidth{0.4pt} % Size of the footer rule

\setlength\parindent{0pt} % Removes all indentation from paragraphs

%----------------------------------------------------------------------------------------
%	CODE INCLUSION CONFIGURATION
%----------------------------------------------------------------------------------------

\definecolor{lineno}{rgb}{0.5,0.5,0.5}
\definecolor{code}{rgb}{0,0.1,0.6}
\definecolor{keyword}{rgb}{0.5,0.1,0.1}
\definecolor{titlebox}{rgb}{0.85,0.85,0.85}
\definecolor{download}{rgb}{0.8,0.1,0.5}
\definecolor{title}{rgb}{0.4,0.4,0.4}
\definecolor{mygrey}{gray}{0.96}

\lstset{
	language=C,
	basicstyle=\ttfamily\small\color{code},
	showspaces=false,
	%showstringspaces=false,
	%numbers=left,
	%firstnumber=1,
	%stepnumber=1,
%	numberfirstline=true,
	%numberstyle=\color{lineno}\sffamily\scriptsize,
	keywordstyle=\color{keyword}\bfseries,
	stringstyle=\itshape,
	%morekeywords={dosync,if},
	deletekeywords={alter},
	%backgroundcolor=\color{mygrey},
	commentstyle=\color{Green},
}

%----------------------------------------------------------------------------------------
%	DOCUMENT STRUCTURE COMMANDS
%	Skip this unless you know what you're doing
%----------------------------------------------------------------------------------------

% Header and footer for when a page split occurs within a problem environment
\newcommand{\enterProblemHeader}[1]{
	\nobreak\extramarks{#1}{#1 continued on next page\ldots}\nobreak
	\nobreak\extramarks{#1 (continued)}{#1 continued on next page\ldots}\nobreak
}

% Header and footer for when a page split occurs between problem environments
\newcommand{\exitProblemHeader}[1]{
	\nobreak\extramarks{#1 (continued)}{#1 continued on next page\ldots}\nobreak
	\nobreak\extramarks{#1}{}\nobreak
}

\setcounter{secnumdepth}{0} % Removes default section numbers
\newcounter{homeworkProblemCounter} % Creates a counter to keep track of the number of problems

\newcommand{\homeworkProblemName}{}
\newenvironment{homeworkProblem}[1][Problem \arabic{homeworkProblemCounter}]{ % Makes a new environment called homeworkProblem which takes 1 argument (custom name) but the default is "Problem #"
	\stepcounter{homeworkProblemCounter} % Increase counter for number of problems
	\renewcommand{\homeworkProblemName}{#1} % Assign \homeworkProblemName the name of the problem
	\section{\homeworkProblemName} % Make a section in the document with the custom problem count
	\enterProblemHeader{\homeworkProblemName} % Header and footer within the environment
}{
	\exitProblemHeader{\homeworkProblemName} % Header and footer after the environment
}

\newcommand{\problemAnswer}[1]{ % Defines the problem answer command with the content as the only argument
\noindent\framebox[\columnwidth][c]{\begin{minipage}{0.98\columnwidth}#1\end{minipage}} % Makes the box around the problem answer and puts the content inside
}

\newcommand{\homeworkSectionName}{}
\newenvironment{homeworkSection}[1]{ % New environment for sections within homework problems, takes 1 argument - the name of the section
	\renewcommand{\homeworkSectionName}{#1} % Assign \homeworkSectionName to the name of the section from the environment argument
	\subsection{\homeworkSectionName} % Make a subsection with the custom name of the subsection
	\enterProblemHeader{\homeworkProblemName\ [\homeworkSectionName]} % Header and footer within the environment
}{
	\enterProblemHeader{\homeworkProblemName} % Header and footer after the environment
}

%----------------------------------------------------------------------------------------
%	NAME AND CLASS SECTION
%----------------------------------------------------------------------------------------

\newcommand{\hmwkTitle}{Assignment\ \#3} % Assignment title
%\newcommand{\hmwkDueDate}{Monday,\ January\ 1,\ 2012} % Due date
\newcommand{\hmwkClass}{ME 766} % Course/class
%\newcommand{\hmwkClassTime}{10:30am} % Class/lecture time
%\newcommand{\hmwkClassInstructor}{Jones} % Teacher/lecturer
\newcommand{\hmwkAuthorName}{110070039} % Your name

%----------------------------------------------------------------------------------------
%	TITLE PAGE
%----------------------------------------------------------------------------------------

\title{
	\vspace{2in}
	\textmd{\textbf{\hmwkClass:\ \hmwkTitle}}\\
	%\normalsize\vspace{0.1in}\small{Due\ on\ \hmwkDueDate}\\
	%\vspace{0.1in}\large{\textit{\hmwkClassInstructor\ \hmwkClassTime}}
	\vspace{3in}
}

\author{\textbf{\hmwkAuthorName}}
\date{} % Insert date here if you want it to appear below your name

%----------------------------------------------------------------------------------------

\begin{document}


%----------------------------------------------------------------------------------------
%	PROBLEM 1
%----------------------------------------------------------------------------------------

% To have just one problem per page, simply put a \clearpage after each problem
Matrix multiplication kernel:
\lstinputlisting[language=C, firstline=89, lastline=119]{../codes/mm_cuda.cu}

In the above code, \verb|BLOCKSIZE| is the number of threads per dimension in
a 2-D square block and \verb|NC| is the size of the matrix.

\textbf{Results}
\begin{table}[h]
	\sisetup{round-mode=places}
	\caption{Time taken. N - matrix size, n - threads/procesess}
	\centering
%	\begin{center}
	\begin{tabular}{| S[round-precision=3] | S[round-precision=3] | S[round-precision=3] | S[round-precision=3] | S[round-precision=3] | S[round-precision=3] |}
			\hline
			{N}&{serial} & {OMP n= 24} & {MPI n=24} & {CUDA(16)} & {CUDA(32)}\\ \hline
			{100}&0.008205
			&0.006868 &0.005075 &0.160000 &0.140000 \\ \hline

			{200}&0.064621&0.010956
			&0.022578&0.170000 &0.140000\\ \hline

			{400} &0.610361&0.080594 
			&0.092229&0.170000 &0.150000\\ \hline

			{800}&4.080133&0.486476
			&0.588597&0.170000 &0.170000\\ \hline

			{1000}&5.573937&0.593531
			&0.717748 &0.200000 &0.200000\\ \hline

			{2000}&45.544228&4.064920
			&4.581032 &0.390000 &0.350000\\ \hline

			{5000}&703.574890&62.474705
			&57.712799 &1.970000 &1.770000\\ \hline

			{8000}&2878.790771 
			&234.618332
			&233.626678 &13.420000 &12.060000 \\ \hline

			{10000}&6591.408691 
			&499.029694&557.603271 &26.030000 &23.440000\\ 
			\hline
		\end{tabular}
%	\end{center}


\end{table}
The following table gives the runtimes of serial, OMP, MPI, and CUDA codes.
\verb|float| datatype is used even though the numbers are large enough
to overflow hence the results(matrix C) are consistent but may not be correct. Also, 
for CUDA programs N used is the nearest multiple of \verb|BLOCKSIZE| to the
number mentioned in the table. \\
\newpage
The following can be inferred from the results:
\begin{itemize}
	\item CUDA implementation outperforms the other three implementations for large N

	\item The runtimes of CUDA code are more or less the same for N$<$1000
	\item Data flow through PCIe becomes the bottleneck hence the runtimes may not
		scale as expected with N.   
	\item CUDA kernel that user 32x32 block size is slightly faster compared 
		to the 16x16 one. This is due to better usage of shared memory.
\end{itemize}
\textbf{Hardware used}
The machine on which the above results were produced has 24 cores, 15MB L3 cache and it is running Debian Server OS.\\
GPU specs.: \\
Name: GeForce GTX 650 Ti \\
Shared memory: 49152 \\
Max threads per block : 1024 \\
Max blocks: 65535 \\
total Const mem: 65536 \\

\newpage
\textbf{Code}
\lstinputlisting{../codes/mm_cuda.cu}
\end{document}
