\documentclass{jhwhw}
\usepackage{libertine}
\author{Roll No.: 110070039}
\title{ME 766 : HW 1}

\begin{document}
\problem
\solution
\textbf{Amdahl's Law} states that if $P$ is the proportion of a program that can be made parallel, then the maximum speedup that can be achieved by using $N$ processors is 
\begin{equation*}
s(N) = \frac{1}{(1-P) + P/N}
\end{equation*} 
\part The maximum achievable speedup is $s(\infty)=1/(1-P)$. Here $P = 4/5$, therefore the maximum speedup is 5.
\part If the desirable speedup is 50, the maximum percentage of the serial portion for the algorithm is 1/50.

\problem
\solution
\textbf{Assumption: } Array A is of type \textit{float}.
\part 1024 B of data in main memory requires (150+$\frac{1024}{8}$ = 278) CPU cycles to process. Array A is of size $1024 \times 1024 \times 8$ Bytes. Therefore the cost to fetch the whole matrix is $278 \times 1024 \times 8 = 2277376$ CPU cycles.
\part Since FORTRAN stores in column-major order, cache needs to be overwritten for each element access. Each element takes 150 CPU cycles for memory fetch and 1 CPU cycle for cache fetch. Therefore the total cost to fetch A to perform the computation is $1024 \times 1024 \times 151 = 158334976$ CPU cycles which is about 70 times the number of CPU cycles required for C or any other row-major order language code. 

\end{document}


